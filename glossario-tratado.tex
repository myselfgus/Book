\chapter{Glossário de Termos Técnicos}
\label{ch:glossario}

\begin{description}

\item[Ação Comunicativa] Conceito desenvolvido por Habermas que se refere a uma forma de interação social na qual os participantes coordenam seus planos de ação através do entendimento mútuo, visando o consenso racional. Na terapia, representa o processo dialógico entre terapeuta e paciente orientado à emancipação.

\item[Adaptatividade] Propriedade de um sistema de ajustar seu comportamento com base em experiências passadas ou feedbacks. No contexto do tratado, refere-se à capacidade do indivíduo ou do terapeuta de modificar abordagens com base no desempenho ou resultados anteriores.

\item[Afeto] Estados emocionais básicos que influenciam a cognição e o comportamento. Na perspectiva de Spinoza, são expressões da potência do ser. No tratado, os afetos são vistos como forças que moldam e são moldadas pela linguagem.

\item[Atrator] Conceito de sistemas dinâmicos que representa um conjunto de estados para o qual o sistema evolui ao longo do tempo. No contexto cognitivo, pode referir-se a padrões estáveis de pensamento ou comportamento para os quais o indivíduo tende a retornar.

\item[Autenticidade do Ser] Estado de existência onde o indivíduo age e se expressa em conformidade com seus valores e natureza interna, livre de imposições externas ou alienações. Objetivo fundamental do processo terapêutico descrito no tratado.

\item[Autonomia] Capacidade de autodeterminação e autorregulação, onde o indivíduo age conforme princípios e valores próprios. Meta central do processo terapêutico, atingida através da compreensão e organização de afetos e cognições pela linguagem.

\item[Coerência Discursiva] Qualidade do discurso que apresenta unidade lógica e consistência entre suas partes. No tratado, é vista como reflexo da organização mental e indicador do estado cognitivo-emocional do paciente.

\item[Cognição] Conjunto de processos mentais envolvidos na aquisição de conhecimento, compreensão e interpretação do mundo. Inclui percepção, atenção, memória, raciocínio, julgamento e tomada de decisão.

\item[Defusão Cognitiva] Técnica terapêutica da ACT (Terapia de Aceitação e Compromisso) que envolve criar distância entre si mesmo e os próprios pensamentos, observando-os sem se identificar completamente com eles ou tomá-los como verdades absolutas.

\item[Discurso] Manifestação verbal do pensamento, estruturada segundo regras linguísticas e influenciada por contextos sociais e psicológicos. No tratado, é analisado como expressão e construtor da subjetividade.

\item[Emoção] Estados afetivos complexos que envolvem mudanças fisiológicas, cognitivas e comportamentais em resposta a estímulos internos ou externos. Diferente dos afetos básicos, as emoções são mais elaboradas e frequentemente culturalmente moduladas.

\item[Estabilidade] Propriedade de um sistema de resistir a perturbações e manter-se em equilíbrio ou retornar a um estado de equilíbrio após perturbações. No contexto cognitivo, refere-se à capacidade de manter coerência e consistência no pensamento e comportamento.

\item[Flexibilidade Psicológica] Capacidade de perceber e se adaptar a situações mutáveis, persistindo ou mudando comportamentos conforme necessário para alcançar objetivos alinhados com valores pessoais. Conceito central da ACT.

\item[Gramática Emocional] Conjunto de regras implícitas que governam a expressão e interpretação de emoções em determinado contexto cultural ou social. Influencia como sentimentos são articulados e compreendidos através da linguagem.

\item[Interdependência] Relação de dependência mútua onde elementos influenciam e são influenciados reciprocamente. No tratado, descreve a relação entre cognição, emoção e linguagem como processos inseparáveis.

\item[Integral] Conceito matemático que representa a soma de infinitesimais ou o acúmulo contínuo. No tratado, usado para descrever como emoções e experiências se acumulam ao longo do tempo, formando o mapa afetivo.

\item[Linguagem] Sistema estruturado de comunicação que utiliza palavras, símbolos e regras gramaticais para expressar significados. No tratado, é vista não apenas como meio de expressão, mas como organizadora do pensamento e ferramenta para a construção do self.

\item[Mapa Afetivo] Representação metafórica da acumulação de estados emocionais ao longo do tempo, formando uma ``topografia'' interna de experiências emocionais. Permite visualizar padrões e trajetórias afetivas.

\item[Metáfora] Figura de linguagem que estabelece uma comparação implícita entre elementos não literalmente relacionados. No tratado, é vista como ferramenta fundamental para expressar emoções complexas e facilitar a compreensão de estados internos.

\item[Narrativa] Relato estruturado de eventos interconectados, com início, meio e fim, que confere significado e coerência às experiências. No tratado, é vista como ferramenta para organizar o caos emocional e construir o self.

\item[Plasticidade] Capacidade de mudar e adaptar-se em resposta a novas experiências ou demandas. No contexto neurobiológico, refere-se à capacidade do cérebro de reorganizar-se; no terapêutico, à capacidade de transformação do paciente.

\item[Reflexividade] Propriedade de voltar-se sobre si mesmo; capacidade de auto-observação e auto-modificação. Na terapia, refere-se à habilidade de examinar os próprios pensamentos, emoções e comportamentos.

\item[Resiliência] Capacidade de recuperar-se ou adaptar-se positivamente após experiências adversas. No contexto terapêutico, relaciona-se à habilidade de manter coerência mental e emocional mesmo diante de perturbações.

\item[Self] Concepção que um indivíduo tem de si mesmo, incluindo aspectos físicos, psicológicos, sociais e existenciais. No tratado, é visto como uma construção narrativa mediada pela linguagem.

\item[Sistemas Dinâmicos] Campo da matemática que estuda o comportamento de sistemas que evoluem no tempo segundo regras fixas. No tratado, utilizado para modelar a evolução dos estados mentais e emocionais.

\item[Subjetividade] Perspectiva individual e interna de experienciar o mundo, influenciada por fatores pessoais, sociais e culturais. No tratado, é vista como construída na interseção entre linguagem, cognição e emoção.

\item[Trajetória Cognitiva] Sequência de estados mentais pelo qual o indivíduo passa ao longo do tempo. No tratado, é analisada através da evolução da linguagem como indicador do progresso terapêutico.

\end{description}
