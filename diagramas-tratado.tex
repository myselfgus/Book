\chapter{Diagramas e Representações Lógico-Matemáticas}
\label{ch:diagramas}

\section{Diagrama 1: Interdependência Mental (Seção 1)}

\subsection{Representação Matemática da Interdependência}

A interdependência entre cognição ($C$), emoção ($E$) e linguagem ($L$) pode ser formalizada como um sistema de equações diferenciais acopladas:

\begin{align}
\frac{dC}{dt} &= f_C(C,E,L)\\[1em]
\frac{dE}{dt} &= f_E(C,E,L)\\[1em]
\frac{dL}{dt} &= f_L(C,E,L)
\end{align}

Onde as funções $f_C$, $f_E$ e $f_L$ representam a taxa de mudança de cada componente em relação aos outros, capturando a interdependência dinâmica do sistema.

\subsection{Representação Visual como Sistema Dinâmico}

O diagrama representa a circularidade e interdependência total do sistema, onde cada elemento influencia e é influenciado pelos outros. A Cognição ($C$) influencia a Linguagem ($L$), que por sua vez influencia a Emoção ($E$), completando o ciclo de volta à Cognição.

\section{Diagrama 2: Mapa Afetivo Longitudinal (Seção 4)}

\subsection{Representação Matemática do Mapa Afetivo}

O estado emocional acumulado ($E_A$) pode ser representado como a integral das expressões emocionais na linguagem ($L_E$) ao longo do tempo:

\begin{equation}
E_A(t) = \int_{t_0}^{t} L_E(\tau) \, d\tau + E_0
\end{equation}

Onde:
\begin{itemize}
\item $E_A(t)$ é o estado emocional acumulado no tempo $t$
\item $L_E(\tau)$ é a expressão emocional na linguagem no tempo $\tau$
\item $E_0$ é o estado emocional inicial
\item $t_0$ é o tempo inicial
\end{itemize}

A taxa de mudança do estado emocional (derivada) seria:

\begin{equation}
\frac{dE_A}{dt} = L_E(t)
\end{equation}

Isto indica que a linguagem emocional atual determina a taxa de mudança do estado emocional acumulado.

\section{Diagrama 3: O Triângulo do Poder, Cultura e Subjetividade (Seção 5)}

\subsection{Representação Matemática do Sistema Triangular}

A relação entre poder ($P$), cultura ($C$) e subjetividade ($S$) pode ser formalizada como um sistema de transformações mutuamente influentes mediadas pela linguagem ($L$):

\begin{equation}
S' = L \circ (P \times C) \circ S
\end{equation}

Onde:
\begin{itemize}
\item $S'$ é a subjetividade transformada
\item $L$ é a função de linguagem que medeia as relações
\item $P \times C$ representa a interação entre poder e cultura
\item $\circ$ denota a composição de funções
\end{itemize}

A linguagem ($L$) ocupa o centro do triângulo, mediando todas as relações entre os vértices Poder, Cultura e Subjetividade.

\section{Diagrama 4: Processo de Reestruturação Narrativa (Seção 6)}

\subsection{Representação Matemática da Reestruturação Narrativa}

A reestruturação narrativa pode ser representada como uma transformação de um estado narrativo fragmentado ($N_f$) para um estado narrativo integrado ($N_i$) através de operadores linguísticos ($L_{op}$):

\begin{equation}
N_i = \prod_{j=1}^{n} L_{op,j} \circ N_f
\end{equation}

Onde:
\begin{itemize}
\item $L_{op,j}$ representa o j-ésimo operador linguístico aplicado na terapia
\item $\prod$ representa a composição sequencial dos operadores
\item $\circ$ denota a operação de aplicação do operador à narrativa
\end{itemize}

Esta transformação pode resultar em mudanças neurais, representadas pela função:

\begin{equation}
\Delta N_{neural} = \phi(N_i - N_f)
\end{equation}

Onde $\phi$ é uma função que mapeia mudanças narrativas para mudanças neurais.

\section{Diagrama 5: Evolução Longitudinal da Linguagem na Terapia (Seção 8)}

\subsection{Representação Matemática da Evolução Terapêutica}

A evolução da coerência linguística ($C_L$) ao longo do tempo pode ser modelada como:

\begin{equation}
C_L(t) = C_L(0) + \int_{0}^{t} r(\tau) \, d\tau
\end{equation}

Onde:
\begin{itemize}
\item $C_L(t)$ é a coerência linguística no tempo $t$
\item $C_L(0)$ é a coerência linguística inicial
\item $r(\tau)$ é a taxa de progresso terapêutico no tempo $\tau$
\end{itemize}

A taxa de progresso pode ser expressa como:

\begin{equation}
r(t) = \alpha \cdot T(t) \cdot P(t) - \beta \cdot R(t)
\end{equation}

Onde:
\begin{itemize}
\item $T(t)$ representa a eficácia da intervenção terapêutica
\item $P(t)$ representa o engajamento do paciente
\item $R(t)$ representa fatores de resistência
\item $\alpha$ e $\beta$ são coeficientes de peso
\end{itemize}

\section{Diagrama 6: Comunicação Terapêutica (Seção 9)}

\subsection{Modelo Matemático da Eficácia Comunicativa}

A eficácia da ação comunicativa terapêutica ($E_{ACT}$) pode ser modelada como:

\begin{equation}
E_{ACT} = f(A_T, R_P, Q_C)
\end{equation}

Onde:
\begin{itemize}
\item $A_T$ representa a autenticidade do terapeuta
\item $R_P$ representa a responsividade do paciente
\item $Q_C$ representa a qualidade da comunicação
\end{itemize}

A qualidade da comunicação pode ser definida como:

\begin{equation}
Q_C = \frac{M_C \cdot E_A}{D_P + 1}
\end{equation}

Onde:
\begin{itemize}
\item $M_C$ é a clareza mútua da comunicação
\item $E_A$ é a empatia ativa do terapeuta
\item $D_P$ representa as distorções na percepção
\end{itemize}

\section{Diagrama 7: Sistema Tripartite de Diagnóstico, Intervenção e Transformação (Seção 10)}

\subsection{Modelo Matemático do Processo Terapêutico Integrado}

O processo terapêutico integrado pode ser representado como um sistema de equações:

\textbf{Diagnóstico:}
\begin{equation}
D = \alpha_1 L_R + \alpha_2 L_C + \alpha_3 L_E
\end{equation}

\textbf{Intervenção:}
\begin{equation}
I = \beta_1 D + \beta_2 T_{comp} + \beta_3 P_{eng}
\end{equation}

\textbf{Transformação:}
\begin{equation}
\Delta S = \gamma_1 I + \gamma_2 N_{reorg} + \gamma_3 \int I(t) dt
\end{equation}

Onde:
\begin{itemize}
\item $L_R$, $L_C$, $L_E$ representam aspectos reflexivos, cognitivos e emocionais da linguagem
\item $T_{comp}$ é a competência terapêutica
\item $P_{eng}$ é o engajamento do paciente
\item $N_{reorg}$ é a reorganização narrativa
\item $\alpha_i$, $\beta_i$, $\gamma_i$ são coeficientes de peso
\item $\Delta S$ é a mudança na subjetividade
\item $\int I(t) dt$ representa o efeito cumulativo da intervenção ao longo do tempo
\end{itemize}

\section{Modelo Axiomático Formal (Conclusão)}

\subsection{Axiomas em Notação Formal}

\textbf{A1 (Interdependência Mental):}
\begin{equation}
\forall x, y \in \{C, E, L\}: \frac{\partial x}{\partial y} \neq 0
\end{equation}

\textbf{A2 (Nomeação Existencial):}
\begin{equation}
\exists(e) \iff \exists N(e)
\end{equation}

\textbf{A3 (Linguagem Reflexiva):}
\begin{equation}
L = f(S_{interno})
\end{equation}

\textbf{A4 (Transformação Linguística):}
\begin{equation}
\Delta S_{interno} = g(\Delta L)
\end{equation}

\textbf{A5 (Ação Comunicativa Emancipatória):}
\begin{equation}
A_{comunicativa} \Rightarrow Emancipa\c{c}\tilde{a}o
\end{equation}

\textbf{A6 (Influência Sociocultural):}
\begin{equation}
L = h(P, C, S)
\end{equation}

\subsection{Teorema da Linguagem Transformadora}

\textbf{Teorema:}
\begin{equation}
Axioma_1 \land Axioma_4 \land Postulado_1 \Rightarrow (Interven\c{c}\tilde{a}o_L \Rightarrow \Delta S_{cognitivo-emocional} \Rightarrow Autonomia)
\end{equation}

\subsection{Corolário}

\begin{equation}
An\acute{a}lise_L \land Reestrutura\c{c}\tilde{a}o_L \Rightarrow \{Terapia, Educa\c{c}\tilde{a}o, Transforma\c{c}\tilde{a}o_{social}\}
\end{equation}
